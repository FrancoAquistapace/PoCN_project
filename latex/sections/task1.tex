\chapter{Song-Havlin-Makse self-similar model}

\resp{Franco Aquistapace Tagua}

\emph{
Structure as\footnote{Remove this part from the report}:
\begin{itemize}
\item A short (max 1 page) explanation of the task, including references. Include mathematical concepts.
\item Max 2 pages for the whole task (including figures)
\item It is possible to use appendices for supplementary material, at the end of the report. Max 5 pages per task
\end{itemize}
A total of 3 pages + 5 supplementary pages per task
}

\section{Introduction and methods}
 
The work by Song, Havlin and Makse aims to explain the characteristic features of empirical fractal and non-fractal networks by means of a single growth model, that has the concept of renormalisation at its core. More specifically, the growth mechanism works as the inverse of the 

\section{Results and discussion}

\lipsum[2-4]


\newpage