\chapter{Data analytics of public transport networks}


\section{Introduction and methods}
 The Citylines platform is a collaborative effort to map the public transport systems of the world. It offers a dataset that contains a detailed history of the development of transport networks in several cities, from multiple countries. The aim of this task is to reconstruct the transport networks from the available data, and to perform an analysis of such networks. The first part of the task is concerned with the development of the methods and heuristics that are necessary to reconstruct and infer networks from the geographical location of nodes and edges. Each network is built as a set of connected components, where each connected component represents a public transport line, in which the nodes represent stations and edges are a simplified representation of the sections connecting the stations. The second part of the task is concerned with the analysis and characterisation of the produced networks, by means of standard topological descriptors.
 
 The 


\section{Results and discussion}



\newpage