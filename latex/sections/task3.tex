\chapter{Data analytics of public transport networks}


\section{Introduction and methods}
 The Citylines platform is a collaborative effort to map the public transport systems of the world \cite{Citylines}. It offers a dataset that contains a detailed history of the development of transport networks in several cities, from multiple countries. The aim of this task is to reconstruct the transport networks from the available data, and to perform an analysis of such networks. The first part of the task is concerned with the development of the methods and heuristics that are necessary to reconstruct and infer networks from the geographical location of nodes and edges. Each network is built as a set of connected components, where each connected component represents a public transport line, in which the nodes represent stations and edges are a simplified representation of the sections connecting the stations. The second part of the task is concerned with the analysis and characterisation of the produced networks, by means of standard topological descriptors.
 
 The construction of the transport network for a given city is done by iterating the same procedure for all available transport lines. First, all stations that are linked to the line are stored as nodes. Then, the edges between the nodes are built in a three--step process. Initially, each node attempts to connect to its two closest neighbours, using their geographical location as position. The connection is accepted unless the neighbours are closer to each other than to the reference node, in which case the edge is only added between the reference node and its closest neighbour. Then, for every section of the line, an edge between the two nodes that are closest to the respective ends is attempted. In this case, the edge is rejected if there already exists a path between the nodes. Finally, a rewiring step is performed to avoid spurious edges: for each two connected nodes, if a shorter edge between one of the nodes and a neighbour is possible then the original edge is replaced by the shorter alternative. This rewiring is repeated until no more replacements are possible. A check is performed throughout the entire process so that the same edge is not built multiple times.


\section{Results and discussion}



\newpage